\PassOptionsToPackage{unicode=true}{hyperref} % options for packages loaded elsewhere
\PassOptionsToPackage{hyphens}{url}
%
\documentclass[11pt,]{article}
\usepackage{lmodern}
\usepackage{amssymb,amsmath}
\usepackage{ifxetex,ifluatex}
\usepackage{fixltx2e} % provides \textsubscript
\ifnum 0\ifxetex 1\fi\ifluatex 1\fi=0 % if pdftex
  \usepackage[T1]{fontenc}
  \usepackage[utf8]{inputenc}
  \usepackage{textcomp} % provides euro and other symbols
\else % if luatex or xelatex
  \usepackage{unicode-math}
  \defaultfontfeatures{Ligatures=TeX,Scale=MatchLowercase}
\fi
% use upquote if available, for straight quotes in verbatim environments
\IfFileExists{upquote.sty}{\usepackage{upquote}}{}
% use microtype if available
\IfFileExists{microtype.sty}{%
\usepackage[]{microtype}
\UseMicrotypeSet[protrusion]{basicmath} % disable protrusion for tt fonts
}{}
\IfFileExists{parskip.sty}{%
\usepackage{parskip}
}{% else
\setlength{\parindent}{0pt}
\setlength{\parskip}{6pt plus 2pt minus 1pt}
}
\usepackage{hyperref}
\hypersetup{
            pdftitle={Title of the semester project},
            pdfborder={0 0 0},
            breaklinks=true}
\urlstyle{same}  % don't use monospace font for urls
\usepackage[margin=1in]{geometry}
\usepackage{color}
\usepackage{fancyvrb}
\newcommand{\VerbBar}{|}
\newcommand{\VERB}{\Verb[commandchars=\\\{\}]}
\DefineVerbatimEnvironment{Highlighting}{Verbatim}{commandchars=\\\{\}}
% Add ',fontsize=\small' for more characters per line
\usepackage{framed}
\definecolor{shadecolor}{RGB}{248,248,248}
\newenvironment{Shaded}{\begin{snugshade}}{\end{snugshade}}
\newcommand{\AlertTok}[1]{\textcolor[rgb]{0.94,0.16,0.16}{#1}}
\newcommand{\AnnotationTok}[1]{\textcolor[rgb]{0.56,0.35,0.01}{\textbf{\textit{#1}}}}
\newcommand{\AttributeTok}[1]{\textcolor[rgb]{0.77,0.63,0.00}{#1}}
\newcommand{\BaseNTok}[1]{\textcolor[rgb]{0.00,0.00,0.81}{#1}}
\newcommand{\BuiltInTok}[1]{#1}
\newcommand{\CharTok}[1]{\textcolor[rgb]{0.31,0.60,0.02}{#1}}
\newcommand{\CommentTok}[1]{\textcolor[rgb]{0.56,0.35,0.01}{\textit{#1}}}
\newcommand{\CommentVarTok}[1]{\textcolor[rgb]{0.56,0.35,0.01}{\textbf{\textit{#1}}}}
\newcommand{\ConstantTok}[1]{\textcolor[rgb]{0.00,0.00,0.00}{#1}}
\newcommand{\ControlFlowTok}[1]{\textcolor[rgb]{0.13,0.29,0.53}{\textbf{#1}}}
\newcommand{\DataTypeTok}[1]{\textcolor[rgb]{0.13,0.29,0.53}{#1}}
\newcommand{\DecValTok}[1]{\textcolor[rgb]{0.00,0.00,0.81}{#1}}
\newcommand{\DocumentationTok}[1]{\textcolor[rgb]{0.56,0.35,0.01}{\textbf{\textit{#1}}}}
\newcommand{\ErrorTok}[1]{\textcolor[rgb]{0.64,0.00,0.00}{\textbf{#1}}}
\newcommand{\ExtensionTok}[1]{#1}
\newcommand{\FloatTok}[1]{\textcolor[rgb]{0.00,0.00,0.81}{#1}}
\newcommand{\FunctionTok}[1]{\textcolor[rgb]{0.00,0.00,0.00}{#1}}
\newcommand{\ImportTok}[1]{#1}
\newcommand{\InformationTok}[1]{\textcolor[rgb]{0.56,0.35,0.01}{\textbf{\textit{#1}}}}
\newcommand{\KeywordTok}[1]{\textcolor[rgb]{0.13,0.29,0.53}{\textbf{#1}}}
\newcommand{\NormalTok}[1]{#1}
\newcommand{\OperatorTok}[1]{\textcolor[rgb]{0.81,0.36,0.00}{\textbf{#1}}}
\newcommand{\OtherTok}[1]{\textcolor[rgb]{0.56,0.35,0.01}{#1}}
\newcommand{\PreprocessorTok}[1]{\textcolor[rgb]{0.56,0.35,0.01}{\textit{#1}}}
\newcommand{\RegionMarkerTok}[1]{#1}
\newcommand{\SpecialCharTok}[1]{\textcolor[rgb]{0.00,0.00,0.00}{#1}}
\newcommand{\SpecialStringTok}[1]{\textcolor[rgb]{0.31,0.60,0.02}{#1}}
\newcommand{\StringTok}[1]{\textcolor[rgb]{0.31,0.60,0.02}{#1}}
\newcommand{\VariableTok}[1]{\textcolor[rgb]{0.00,0.00,0.00}{#1}}
\newcommand{\VerbatimStringTok}[1]{\textcolor[rgb]{0.31,0.60,0.02}{#1}}
\newcommand{\WarningTok}[1]{\textcolor[rgb]{0.56,0.35,0.01}{\textbf{\textit{#1}}}}
\usepackage{graphicx,grffile}
\makeatletter
\def\maxwidth{\ifdim\Gin@nat@width>\linewidth\linewidth\else\Gin@nat@width\fi}
\def\maxheight{\ifdim\Gin@nat@height>\textheight\textheight\else\Gin@nat@height\fi}
\makeatother
% Scale images if necessary, so that they will not overflow the page
% margins by default, and it is still possible to overwrite the defaults
% using explicit options in \includegraphics[width, height, ...]{}
\setkeys{Gin}{width=\maxwidth,height=\maxheight,keepaspectratio}
\setlength{\emergencystretch}{3em}  % prevent overfull lines
\providecommand{\tightlist}{%
  \setlength{\itemsep}{0pt}\setlength{\parskip}{0pt}}
\setcounter{secnumdepth}{0}
% Redefines (sub)paragraphs to behave more like sections
\ifx\paragraph\undefined\else
\let\oldparagraph\paragraph
\renewcommand{\paragraph}[1]{\oldparagraph{#1}\mbox{}}
\fi
\ifx\subparagraph\undefined\else
\let\oldsubparagraph\subparagraph
\renewcommand{\subparagraph}[1]{\oldsubparagraph{#1}\mbox{}}
\fi

% set default figure placement to htbp
\makeatletter
\def\fps@figure{htbp}
\makeatother

\usepackage{longtable}
\usepackage{hologo}
\LTcapwidth=.95\textwidth
\linespread{1.05}
\usepackage{hyperref}

\title{Title of the semester project}
\author{true \and true \and true}
\date{May 18, 2021}

\begin{document}
\maketitle

\hypertarget{rmarkdown-basics}{%
\section{\texorpdfstring{\texttt{RMarkdown}
basics}{RMarkdown basics}}\label{rmarkdown-basics}}

This is a citations: Efron (1992).

This is a displayed but not evaluated \texttt{R} code chunk

\begin{Shaded}
\begin{Highlighting}[]
\KeywordTok{print}\NormalTok{(}\StringTok{"I love R"}\NormalTok{)}
\end{Highlighting}
\end{Shaded}

This is an \texttt{R} code chunk, not displayed but evaluated.

\begin{center}\includegraphics{README_files/figure-latex/unnamed-chunk-2-1} \end{center}

This is an inline \texttt{R} code: Hence, the mean of the data is of
-0.0247732.

This is a \hologo{LaTeX} equation

\[
f(x)=\frac{1}{\sigma \sqrt{2 \pi}} \mathrm{e}^{-\frac{1}{2}\left(\frac{x-\mu}{\sigma}\right)^{2}}
\]

This is a inline \hologo{LaTeX} equation:
\(\frac{1}{n} \sum_{i=1}^{n} a_{i}=\frac{a_{1}+a_{2}+\cdots+a_{n}}{n}\)

\hypertarget{introduction}{%
\section{Introduction}\label{introduction}}

La moyenne vaut -0.02

\hypertarget{analysis}{%
\section{Analysis}\label{analysis}}

\hypertarget{description-of-the-task}{%
\section{Description of the task}\label{description-of-the-task}}

\hypertarget{motivation}{%
\section{Motivation}\label{motivation}}

\hypertarget{results-description-and-interpretation}{%
\section{Results: description and
interpretation}\label{results-description-and-interpretation}}

\hypertarget{parameters}{%
\section{Parameters}\label{parameters}}

SAMPLESEED \textless{}- 9886 SIMULATIONSEED \textless{}- 1021 n
\textless{}- 500 muX \textless{}- 0 muY \textless{}- 1 sX \textless{}-
1.8 sY \textless{}- 0.5 rho \textless{}- 0.44 out \textless{}- 0.07 dev
\textless{}- 3 angle \textless{}-0

\hypertarget{generate-data-nonlinear-setting}{%
\section{Generate data, nonlinear
setting}\label{generate-data-nonlinear-setting}}

n \textless{}- 500 set.seed(SAMPLESEED) df \textless{}-
gen\_nonlinear(n=n, muX=muX, muY=muY, sX=sX, sY=sY, angle=pi) \#
population correlation in this setting rho \textless{}-
cor\_nonlinear(muX=muX, muY=muY, sX=sX, sY=sY, angle=angle) plot(df,
main=paste(``Population correlation ='', round(rho, 3)))

\hypertarget{were-these-results-expected-discussion}{%
\section{Were these results expected:
discussion}\label{were-these-results-expected-discussion}}

\hypertarget{statistical-methods-used}{%
\section{Statistical methods used}\label{statistical-methods-used}}

\hypertarget{acquired-skills-during-the-term-project}{%
\section{Acquired skills during the term
project}\label{acquired-skills-during-the-term-project}}

\hypertarget{additional-element}{%
\section{Additional element}\label{additional-element}}

\hypertarget{conclusion}{%
\section*{Conclusion}\label{conclusion}}
\addcontentsline{toc}{section}{Conclusion}

\hypertarget{refs}{}
\leavevmode\hypertarget{ref-efron1992bootstrap}{}%
Efron, Bradley. 1992. ``Bootstrap Methods: Another Look at the
Jackknife.'' In \emph{Breakthroughs in Statistics}, 569--93. Springer.

\end{document}
